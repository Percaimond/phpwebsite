
%%% Preamble
\documentclass{article}

% For encoding of special chars
\usepackage[utf8]{inputenc}
% For German language and hyphenation
\usepackage[german]{babel}
% For figures
\usepackage{graphicx}
% For quotation marks
\usepackage{csquotes}

%%% Maketitle metadata
\title{
		\vspace{-3cm}
		\normalfont
		\normalsize
		\textsc{Software Requirements Specification (SRS)}\\
		\bigskip
		\huge{Pflichtenheft}
}

% Keep this to prevent date at wrong location
\date{\vspace{-7ex}}

% Keep this to prevent author at wrong location
\author{}

\begin{document}

\maketitle

\noindent\textbf{Softwaresystem:} \textless Name des Systems\textgreater\\
\noindent\textbf{Dokumentversion:} 1.0.0\\
\noindent\textbf{Zuletzt bearbeitet:} \textless
Datum\textgreater\\

\noindent Team \textless Team-ID\textgreater\\
\noindent \textless Vorname1 Nachname1\textgreater, \textless Vorname2 Nachname2\textgreater, \textless Vorname3 Nachname3\textgreater, \textless Vorname4 Nachname4\textgreater, \textless Vorname5 Nachname5\textgreater


\section{Einleitung}
\textless Zweck des Dokuments\textgreater\\
\textless Kurze Beschreibung des Einsatzumfelds des Systems\textgreater\\
\textless Erklärung wichtiger Begriffe (z.B. Buch, Buchkopie und Kunde)\textgreater



\section{Allgemeine Beschreibung}
\textless Kurze Beschreibung des groben Systemumfangs, den Benutzerrollen, mit welchen Technologien es implementiert wird, ob es Schnittstellen mit anderen Systemen gibt und wie mit dem System interagiert wird\textgreater

\subsection{Domain Model}
\textless Bild des Domain Models (Class Diagram)\textgreater\\
\textless Kurze textuelle Beschreibung des Domain Models\textgreater

% Example for embedding a figure
% \begin{figure}
%     \centering
%     \includegraphics[width=\textwidth]{image.jpg}
%     \caption{Beschreibender Bild-Untertitel}
%     \label{fig:label}
% \end{figure}

\subsection{Use Case Diagram}
\textless Bild des Domain Models (Class Diagram)\textgreater\\
\textless Kurze textuelle Beschreibung des Domain Models\textgreater



\section{Spezifische Anforderungen}

\subsection{Funktionale Anforderungen}
\textless Auflistung und Erklärung der 10 Kernprozesse mit wichtigen Details, die für eine Implementierung notwendig sind; für Nr. 8 und 9 reicht ein Verweis auf die entsprechende Use Case Table\textgreater\\\\
\textless Use Case Table \enquote{Ausleihe einer Buchkopie}\textgreater\\
\textless Use Case Table \enquote{Rückgabe einer Buchkopie}\textgreater


\subsection{Nicht-Funktionale Anforderungen}
\textless Auflistung und Beschreibung wichtiger nicht-funktionaler Anforderungen\textgreater 


\end{document}